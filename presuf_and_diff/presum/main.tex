%----------------------------------------------------------------------------------------
%	PACKAGES AND THEMES
%----------------------------------------------------------------------------------------
\documentclass[aspectratio=169,xcolor=dvipsnames]{beamer}
\usetheme{SimplePlus}

\usepackage{hyperref}
\usepackage{graphicx} % Allows including images
\usepackage{booktabs} % Allows the use of \toprule, \midrule and \bottomrule in tables
\usepackage[UTF8]{ctex}
\usepackage{url}
\usepackage{listings}
\usepackage{graphics}
\usepackage{mathtools}
\usepackage{booktabs}
%----------------------------------------------------------------------------------------
%	TITLE PAGE
%----------------------------------------------------------------------------------------

\title[]{前缀和 \&\& 差分} % The short title appears at the bottom of every slide, the full title is only on the title page
% \subtitle{Subtitle}

\author[Ziqi Zhao] {zziqi229@gmail.com}

\institute[NTU] % Your institution as it will appear on the bottom of every slide, may be shorthand to save space
{
    % Department of Computer Science and Information Engineering \\
    % National Taiwan University % Your institution for the title page
}
\date{\today} % Date, can be changed to a custom date


%----------------------------------------------------------------------------------------
%	PRESENTATION SLIDES
%----------------------------------------------------------------------------------------

\begin{document}
\begin{frame}
    \titlepage
\end{frame}

\begin{frame}{Overview}
    \tableofcontents
\end{frame}


\section{开始之前}
\subsection{下标}
\begin{frame}{关于起始下标}

    \begin{itemize}
        \item 在大多数编程语言中数组下标从0开始\\
              例如\text{$w:[10,11,12,13,14,15]$}中$w[0]=10  w[1]=11$

        \item 在算法竞赛中有时为了方便跳过0下标从1开始\\
              例如存储6个数字\text{$10,11,12,13,14,15$}\\
              $w[0]$闲置不使用,$w[1]=10$为实际存储的第一个数字

        \item 根据便利和个人习惯选择起始下标
    \end{itemize}

\end{frame}

\subsection{集合 与 区间}
\begin{frame}
    \frametitle{集合 与 区间}
    
    \textbf{集合:}由一个或多个元素构成的整体,元素的个数可以是有限的也可以是无限的
    \begin{itemize}
        \item 大于$0$小于$10$的奇数组成的集合:$\{1,3,5,7,9\}$
        \item 质数集合:$\{2,3,5,7,11,13,\cdots\}$
        \item 所有正数组成的集合
        \item $\cdots$
    \end{itemize}
    \textbf{区间:}如果x和y是两个在集合里的数,任何x和y之间的数也属于该集合
    \begin{itemize}
        \item $[2,7]$表示所有大于等于2小于等于7的数字组成的集合
        \item $[2,7)$表示所有大于等于2小于7的数字组成的集合
        \item $(2,7]$表示所有大于2小于等于7的数字组成的集合
        \item $(2,7)$表示所有大于2小于7的数字组成的集合
    \end{itemize}
\end{frame}


\section{前缀和}
\begin{frame}{\href{https://www.luogu.com.cn/problem/P8218}{luoguP8218 求区间和}}

    给定 $n$ 个正整数组成的数列 $a_1, a_2, \cdots, a_n$ 和 $m$ 个区间 $[l_i,r_i]$,
    分别求这 $m$ 个区间的区间和。对于所有测试数据,$n,m\le10^5,a_i\le 10^4$\\
    \textbf{输入格式:}\\
    第一行,为一个正整数 $n$ 。\\
    第二行,为 $n$ 个正整数 $a_1,a_2, \cdots ,a_n$\\
    第三行,为一个正整数 $m$ 。\\
    第 $4$ 到第 $n+m+2$ 行,每行为两个正整数 $l_i,r_i$ ,满足$1\le l_i\le r_i\le n$\\
    \textbf{输出格式:}\\
    共 $m$ 行。第 $i$ 行为第 $i$ 组答案的询问。

\end{frame}

\begin{frame}{Solution}
    约定:所有数组以下标1起始\\[20pt]
    $s[i]$为$a$中前$i$个数字的和,即$s[i]=a[1]+a[2]+...+a[i]$\\

    如何计算出$s[i]$
    % \begin{block}{}
    %     for (int i = 1; i <= n; i++)\\
    %     \qquad s[i] = s[i - 1] + a[i];
    % \end{block}
    \begin{block}{Python}
        for i in range(1, n + 1):\\
        \quad s[i] = s[i - 1] + a[i]
    \end{block}

    对于询问$[l,r]$,可以通过$s$得到:\\
    $$ans=s[r]-s[l-1]$$

\end{frame}

\begin{frame}{\href{https://www.luogu.com.cn/problem/P1115}{luoguP1115 最大子段和}}

    给出一个长度为 $n(n \le 2 \times 10^5)$ 的序列 $a(-10^4 \le a_i \le 10^4)$,
    选出其中连续且非空的一段使得这段和最大。\\
    \textbf{输入格式:}\\
    第一行是一个整数,表示序列的长度 $n$。\\
    第二行有 $n$ 个整数,第 $i$ 个整数表示序列的第 $i$ 个数字 $a_i$。\\
    \textbf{输出格式:}\\
    输出一行一个整数表示答案。

\end{frame}

\begin{frame}{Solution}

    约定:不使用a[0],从下标1开始\\

    枚举每一个区间$[l,r](l \le r)$使区间和$a[l]+a[l+1]+...+a[r]$最大\\
    % \begin{block}{}
    %     for(int l=1;l<=n;l++)\{\\
    %     \qquad int sum=0;\\
    %     \qquad for(int r=l;r<=n;r++){\\
    %     \qquad \qquad sum+=a[r];\\
    %     \qquad \qquad ans=max(ans,sum);\\
    %     \qquad }\\
    %     \}
    % \end{block}
    \begin{block}{Python}
        for l in range(1,n+1):\\
        \quad s=0\\
        \quad for r in range(l,n+1):\\
        \quad \quad s+=a[r]\\
        \quad \quad ans=max(ans,s)\\
    \end{block}

\end{frame}

\begin{frame}{Solution}

    计算前缀和$s$,$s[i]=a[1]+a[2]+...+a[i]$\\
    对于某段区间$[l,r](l \le r)$区间和为$s[r]-s[l-1]$\\
    本题即为寻找一对$l,r$使$s[r]-s[l-1]$最大

    % \begin{block}{}
    %     for (int l = 1; l <= n; l++)\\
    %     \qquad for (int r = l; r <= n; r++)\\
    %     \qquad \qquad ans = max(ans, s[r] - s[l - 1]);\\
    % \end{block}

    \begin{block}{Python}
        for l in range(1,n+1):\\
        \quad for r in range(l,n+1):\\
        \quad \quad ans=max(ans,s[r]-s[l-1])\\
    \end{block}
\end{frame}

\begin{frame}{Solution}

    \begin{itemize}
        \item 如果固定$r$,即针对一个$r$,寻找$l$使$s[r]-s[l-1]$最大\\
        \item 即当$s[l-1]$最小时,$s[r]-s[l-1]$最大\\
        \item $s[r]-s[l-1]$最大为$s[r]-min(s[0],s[1],s[2],...,s[r-1])$
    \end{itemize}


    % \begin{block}{}
    %     for (int r = 1; r <= n; r++) \{\\
    %     \qquad int mi = s[0];\\
    %     \qquad for (int i = 0; i <= r - 1; i++) // premin[r-1]\\
    %     \qquad \qquad mi = min(mi, s[i]);\\
    %     \qquad ans = max(ans, s[r] - mi);\\
    %     \}
    % \end{block}

    \begin{block}{Python}
        for r in range(1,n+1)\\
        \quad mi = s[0]\\
        \quad for i in range(r): \# premin[r-1]\\
        \quad \quad mi = min(mi, s[i])\\
        \quad ans = max(ans, s[r] - mi)
    \end{block}
    % 还是会超时

\end{frame}

\begin{frame}{Solution}
    仿照前缀和计算$s$的前缀最小值 $premin[i]=min(s[0],s[1],s[2],...,s[i])$\\
    整体解法为枚举从$1$到$n$枚举$r$,\\
    确定$r$后通过前缀$min$计算最小的$s[r]-s[l-1]$
\end{frame}

\begin{frame}{Solution}
    简化的写法
    \begin{itemize}
        \item 如果固定$r$,即针对一个$r$,寻找$l$使$s[r]-s[l-1]$最大\\
        \item 即当$s[l-1]$最小时,$s[r]-s[l-1]$最大\\
        \item $s[r]-s[l-1]$最大为$s[r]-min(s[0],s[1],s[2],...,s[r-1])$
    \end{itemize}


    \begin{block}{}
        int mi = s[0];\\
        for (int r = 1; r <= n; r++) \{\\
        \qquad ans = max(ans, s[r] - mi);\\
        \qquad mi = min(mi, s[r]);\\
        \}
    \end{block}

\end{frame}



\section{差分}
\begin{frame}{\href{https://www.luogu.com.cn/problem/P2367}{luoguP2367 语文成绩}}
    有$n$名学生,每次需要给第$x$个到第$y$个学生每人增加$z$分。求所有分数更改完毕后全班最低分。\\
    \textbf{输入格式:}\\
    第一行有两个整数 $n (n \le 1 \times 10^6)$,$p(p \le n)$,代表学生数与增加分数的次数。\\
    第二行有 $n$ 个数,$a_1 \sim a_n$,代表各个学生的初始成绩。\\
    接下来 $p$ 行,每行有三个数,$x$,$y$,$z$,代表给第 $x$ 个到第 $y$ 个学生每人增加 $z$ 分。\\
    \textbf{输出格式:}\\
    输出仅一行,代表更改分数后,全班的最低分。
\end{frame}

\begin{frame}{Solution}

    暴力,时间复杂度$O(np)$\\
    % \begin{block}{}
    %     for (int i = x; i <= y; i++)\\
    %     \qquad a[i] += z;
    % \end{block}
    \begin{block}{Python}
        for i in range(x, y + 1)\\
        \quad a[i] += z
    \end{block}
    每次修改的时间复杂度是$O(n)$

\end{frame}

\begin{frame}{差分}

    令$C[i]=a[i]-a[i-1]$\\[20pt]

    如果$a[x],a[x+1],...,a[y]$全部加上$z$\\
    $C$会如何变化?<br>\\[20pt]

    $C[x-1]=a[x-1]-a[x-2]$不变\\
    \alert{$C[x]=a[x]-a[x-1]$加上$z$}\\
    $C[x+1]=a[x+1]-a[x]$不变\\
    ... \\
    $C[y-1]=a[y-1]-a[y-2]$不变\\
    $C[y]=a[y]-a[y-1]$不变\\
    \alert{$C[y+1]=a[y+1]-a[y]$减去$z$}\\
    $C[y+2]=a[y+2]-a[y+1]$不变

\end{frame}

\begin{frame}{Solution}

    对于$a[x],a[x+1],...,a[y]$全部加上$z$\\
    相当于$C[x]+=z,C[y+1]-=z$\\

    $O(n)$
    % \begin{block}{}
    %     for (int i = x; i <= y; i++)\\
    %     \qquad a[i] += z;
    % \end{block}
    \begin{block}{Python}
        for i in range(x, y + 1):\\
        \quad a[i] += z;
    \end{block}
    转换为

    $O(2)$
    % \begin{block}{}
    %     C[x] += z;\\
    %     C[y + 1] -= z;
    % \end{block}
    \begin{block}{Python}
        C[x] += z\\
        C[y + 1] -= z
    \end{block}

\end{frame}

\begin{frame}{Solution}
    但是现在只是将修改记录在了$C$上,
    如何将$C$转换为$a$?
    \begin{align*}
        C[1] & =a[1]        \\
        C[2] & =a[2]-a[1]\\
        C[3] & =a[3]-a[2]\\
        \cdots\\
        C[i] & =a[i]-a[i-1]
    \end{align*}
    将上面的式子相加,得到:
    $$
        a[i]=C[1]+C[2]+...+C[i]
    $$
    $C$做前缀和就可得到$a$

\end{frame}


\begin{frame}{\href{https://www.luogu.com.cn/problem/P3406}{luoguP3406 海底高铁}}

    \href{https://www.luogu.com.cn/problem/P3406}{luoguP3406 海底高铁}

\end{frame}

\section{二维版本}
\subsection{二维前缀和}
\begin{frame}{\href{https://www.luogu.com.cn/problem/P1719}{luoguP1719 最大加权矩形}}
    给出一个$n \times n$的矩阵,矩阵中的每个元素为一个整数.从中找出一个子矩形,使得包含的元素和最大.\\
    $1 \le n \le 120$

    \begin{alertblock}{}
        二维前缀和
        $s[i][j]$表示第一行第一列到第$i$行第$j$列这个矩形的和
    \end{alertblock}
\end{frame}


\subsection{二维差分}
\begin{frame}
    \frametitle{luoguP3397地毯}

    \href{https://www.luogu.com.cn/problem/P3397}{luoguP3397地毯}

\end{frame}

\begin{frame}
    \frametitle{题集}
    二维差分
    \begin{itemize}
        \item P6070
    \end{itemize}

\end{frame}

\end{document}
